\def\DevnagVersion{2.17}%@dollars
\chapter{Preface to the Second Edition}

\textsc{The} principal alterations in the new edition of my Sanskrit
grammar consist in a number of additional references to Pāṇini, in all
cases where an appeal to his authority seemed likely to be useful, and
in the introduction of the marks of the accent. I have also been able to
remove a number of mistakes and misprints which, in spite of all the
care I had taken, had been overlooked in the first edition. Most of
these I had corrected in the German translation of my grammar, published
at Leipzig in 1868; some more have now been corrected. I feel most
grateful to several of my reviewers for having pointed out these
oversights, and most of all to Pandit Rājārāmaśāstrī, whose list of
notes and queries to my grammar has been of the greatest value to me. It
seems almost hopeless for a European scholar to acquire that familiarity
with the intricate system of Pāṇini which the Pandits of the old school
in India still possess; and although some of their refinements in the
interpretation of Pāṇini's rules may seem too subtle, yet there can be
no doubt that these living guides are invaluable to us in exploring the
gigantic labyrinth of ancient Sanskrit grammar.

There is, however, one difficulty which we have to contend with, and
which does not exist for them. They keep true throughout to one system,
the system of Pāṇini; we have to transfer the facts of that system into
our own system of grammar. What accidents are likely to happen during
this process I shall try to illustrate by one instance. Rājārāmaśāstrī
objects to the form {\dn \7{p}\306w\7{s}} \tl{punsu} as the locative plural of
{\dn \7{p}mA\qq{n}} \tl{pumān}. From his point of view, he is perfectly right in
his objection, for according to Pāṇini the locative plural has anusvāra,
{\dn \7{p}\2\7{s}} \tl{puṁsu}. But in our own Sanskrit grammars we first have a
general rule that {\dn \qq{s}} \tl{s} is changed to {\dn \qq{q}} \tl{ṣ} after any vowel
except {\dn a} and {\dn aA} \tl{\v{\={a}}}, in spite of intervening anusvāra
(see \S~100); and it has even been maintained that there is some kind of
physiological reason for such a change. If then, after having laid down
this rule, we yet write {\dn \7{p}\2\7{s}} \tl{puṁsu}, we simply commit a
grammatical blunder; and I believe there is no Sanskrit grammar, except
Colebrooke's, in which that blunder has not been committed. In order to
avoid it, I wrote {\dn \7{p}\306w\7{s}} \tl{punsu}, thus, by the retention of the
dental {\dn \qq{n}} \tl{n}, making it grammatically and physically possible for
the {\dn \qq{s}} \tl{s} to remain unchanged. It may be objected that on the same
ground I ought to have written instrumental {\dn \7{p}\306wsA} \tl{punsā},
genitive {\dn \7{p}\306ws,} \tl{punsaḥ}, \&c.; but in these cases the {\dn \qq{s}} \tl{s}
is radical, and would therefore not be liable to be changed into {\dn \qq{q}}
\tl{ṣ} after a vowel and anusvāra \panini{8.3.59}. Professor Weber had
evidently overlooked these simple rules, or he would have been less
forward in blaming Dr Keller for having followed my example in writing
{\dn \7{p}\306w\7{s}} \tl{punsu}, instead of {\dn \7{p}\2\7{s}} \tl{puṁsu}. In Pāṇini's grammar
(as may be seen from my note appended to \S~100) the rule on the change
of {\dn \qq{s}} \tl{s} into {\dn \qq{q}} \tl{ṣ} is so carefully worded that it just
excludes the case of {\dn \7{p}\2\7{s}} \tl{puṁsu}, although the {\dn \7{s}} \tl{su} of
the locative plural is preceded by an anusvāra. I have now, by making in
my second edition the same reservation in the general rule, been able to
conform to Pāṇini's authority, and have written {\dn \7{p}\2\7{s}} \tl{puṁsu},
instead of {\dn \7{p}\306w\7{s}} \tl{punsu}, though even thus the fact remains that if
the dot is really meant for anusvāra, and if the {\dn \7{s}} \tl{su} is the
termination of the locative plural, the {\dn \qq{s}} \tl{s} would be sounded as
{\dn \qq{q}} \tl{ṣ}, according to the general tendency of the ancient Sanskrit
pronunciation.

I have mentioned this one instance in order to show the peculiar
difficulties which the writer of a Sanskrit grammar has to contend with
in trying to combine the technical rules of Pāṇini with the more
rational principles of European grammar; and I hope it may convince my
readers, and perhaps even Professor Weber, that where I have deviated
from the ordinary rules of our European grammars, or where I seem to
have placed myself at variance with some of the native authorities, I
have not done so without having carefully weighed the advantages of the
one against those of the other system.

F.\@ MAX MÜLLER.

PARKS END, OXFORD,

August, 1870.