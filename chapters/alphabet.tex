\def\DevnagVersion{2.17}%@dollars
\chapter{The Alphabet}

\s Sanskrit is properly written with the Devanāgarī alphabet; but the
Bengali, Telugu, and other modern Indian alphabets are commonly employed
for writing Sanskrit in their respective provinces.

\begin{note}
  Note—\tl{Devaṅāgarī} means the \tl{nāgarī} of the gods, or, possibly,
  of the Brāhmans. A more current style of writing, used by Hindus in
  all common transactions where Hindi is the language employed is called
  simply \tl{nāgarī}. Why the alphabet should have been called
  \tl{nāgarī} is unknown. If derived from \tl{nagara}, \eng{city}, it
  might mean the art of writing as first practised in cities (\panini{}
  iv 2, 128). No authority has yet been adduced from any ancient author
  for the employment of the word \tl{Devanāgarī}. In the
  \emph{Lalitavistara} (a life of Buddha, translated from Sanskrit into
  Chinese 76 AD), where a list of alphabets is given, the
  \tl{Devanāgarī} is not mentioned, unless it be intended by the
  \tl{Deva} alphabet. (See \emph{History of Ancient Sanskrit
    Literature}, p.\ 518.) Al-Biruni, in the 11th century, speaks of the
  \tl{Nagara} alphabet as current in Malva (Reinaud, \emph{Mémoire sur
    l'Inde}, p.\ 298).

  No inscriptions have been met with in India anterior to the rise of
  Buddhism. The earliest authentic specimens of writing as the
  inscriptions of king Priyadarśi or Aśoka, about 250 BC. These are
  written in two different alphabets. The alphabet which is found in the
  inscription of Kapurdigiri, and which in the main is the same as that
  of the Arianian coins, is written from right to left. It is clearly of
  Semitic origin, and most closely connected with the Aramaic branch of
  the old Semitic or Phoenician alphabet. The Aramaic letters, however,
  which we know from Egyptian and Palmyrenian inscriptions, have
  experienced further changes since they served as the model for the
  alphabet of Kapurdigiri, and we must have recourse to the more
  primitive types of the ancient Hebrew coins and of the Phoenician
  inscriptions in order to explain some of the letters of the
  Kapurdigiri alphabet.

  But while the transition of the Semitic types into this ancient Indian
  alphabet can be proved with scientific precision, the second Indian
  alphabet, that which is found in the inscription of Girnar, and which
  is the real source of all other Indian alphabets, as well as of those
  of Tibet and Burma, has not as yet been traced back in a satisfactory
  manner to any Semitic prototype (Prinsep's \emph{Indian Antiquities by
  Thomas}, vol.\ ii, p.\ 42). To admit, however, the independent
  invention of a native Indian alphabet is impossible. Alphabets were
  never invented, in the usual sense of that word. They were formed
  gradually, and purely phonetic alphabets always point back to earlier,
  syllabic or ideographic, stages. There are no such traces of the
  growth of an alphabet on Indian soil; and it is to be hoped that new
  discoveries may still bring to light the intermediate links by which
  the alphabet of Girnar, and through it the modern Devanāgarī, may be
  connected with one of the leading Semitic alphabets.
\end{note}

\s Sanskrit is written from left to right.

\begin{note}
  Note—\tl{Saṁskṛta} ({\dn s\2-\9{k}t}) means what is rendered \eng{fit} or
  \tl{perfect}. But \eng{Sanskrit} is not called so because the
  Brāhmans, or still less, because the first Europeans who became
  acquainted with it, considered it the most perfect of all languages.
  \tl{Saṁskṛta} meant what is rendered fit for sacred purposes; hence
  \eng{purified}, \eng{sacred}. A vessel that is purified, a sacrificial
  victim that is properly dressed, a man who has passed through all the
  initiatory rites or \tl{saṁskāras}; all these are called
  \tl{saṁskṛta}. Hence the language which alone was fit for sacred acts,
  the ancient idiom of the Vedas, was called \tl{Saṁskṛta}, or the
  sacred language. The local spoken dialects received the general name
  of \tl{prākṛta}. This did not mean originally \eng{vulgar}, but
  \eng{derived}, \eng{secondary}, \eng{second-rate}, literally `what has
  a source or type,' this source or type (\tl{prakṛti}) being the
  Saṁskṛta or sacred language. (See Vararuci's \emph{Prākṛta-Prakāśa},
  ed.\ Cowell, p.\ xvii.)
\end{note}

\s In writing the Devanāgarī alphabet, the distinctive portion of each
letter is written first, then the perpendicular, and lastly the
horizontal line. Ex. XXX, XXX, {\dn k} \tl{k}; XXX, XXX, {\dn K} \tl{kh};
XXX, XXX, {\dn g} \tl{g}; XXX, XXX, {\dn G} \tl{gh}; XXX, {\dn R} \tl{ṅ}, \&c.

Beginners will find it useful to trace the letters on transparent paper
till they know them well and can write them fluently and correctly.

\s The following are the sounds which are represented in the Devanāgarī
alphabet:

% TODO right curly brackets in diphthong column
\begin{widepage}
  \begin{tabular}{p{1.6cm}C{1.1cm}C{1.5cm}C{1.1cm}C{1.5cm}C{0.9cm}C{1.1cm}C{1.3cm}C{1cm}C{1cm}C{0.9cm}C{1.1cm}}
    \toprule
    & \footnotesize{Hard (\emph{tenues})} & \footnotesize{Hard and
      aspirated (\emph{tenues aspiratæ})} & \footnotesize{Soft
      (\emph{mediæ})} & \footnotesize{Soft and aspirated (\emph{mediæ
        aspiratæ})} & \footnotesize{Nasal} & \footnotesize{Liquid} &
    \footnotesize{Sibilant} & \multicolumn{2}{c}{\footnotesize{Vowel}} &
    \multicolumn{2}{c}{\footnotesize{Diphthong}}\\
    & & & & & & & & \footnotesize{Short} & \footnotesize{Long}\\
    \midrule
    Gutturals & {\dn k} \tl{k} & {\dn K} \tl{kh} & {\dn g} \tl{g} & {\dn G}
    \tl{gh} & {\dn R} \tl{ṅ} & {\dn h} \tl{h}\footnotemark[2] & X\footnotemark[4]
    (\tl{X}) & {\dn a} \tl{a} & {\dn aA} \tl{ā} & {\dn e} \tl{e} & {\dn e\?} \tl{āi}\\
    Palatals & {\dn c} \tl{c} & {\dn C} \tl{ch} & {\dn j} \tl{j} & {\dn J} \tl{jh} &
    {\dn \31Aw} \tl{ñ} & {\dn y} \tl{y} & {\dn f} \tl{ś} & {\dn i} \tl{i} & {\dn I} \tl{ī}\\
    Linguals & {\dn V} \tl{ṭ} & {\dn W} \tl{ṭh} & {\dn X} \tl{ḍ}\footnotemark[1] & {\dn Y}
    \tl{ḍh}\footnotemark[1] & {\dn Z} \tl{ṇ} & {\dn r} \tl{r} & {\dn q} \tl{ṣ} &
    {\dn \31Bw} \tl{ṛ} & {\dn \311w} \tl{\d{\={r}}} & {\dn ao} \tl{o} & {\dn aO} \tl{au}\\
    Dentals & {\dn t} \tl{t} & {\dn T} \tl{th} & {\dn d} \tl{d} & {\dn D} \tl{dha} &
    {\dn n} \tl{na} & {\dn l} \tl{l} & {\dn s} \tl{s} & {\dn \318w} \tl{ḷ} & ({\dn \319w}
    \tl{\d{\={l}}})\\
    Labials & {\dn p} \tl{p} & {\dn P} \tl{ph} & {\dn b} \tl{b} & {\dn B} \tl{bh} &
    {\dn m} \tl{m} & {\dn v} \tl{v}\footnotemark[3] & X\footnotemark[4] \tl{X}
    & {\dn u} \tl{u} & {\dn U} \tl{ū}\\
    \bottomrule
  \end{tabular}
\end{widepage}

\footnotetext[1]{In the Veda {\dn X} \tl{ḍ} and {\dn Y} \tl{ḍh}, if between
  two vowels, are in certain schools written {\dn \30Fw} \tl{ḷ} and {\dn \310wh} \tl{ḷh}.}
\footnotetext[2]{{\dn h} \tl{h} is not properly a liquid, but a soft
  breathing.}
\footnotetext[3]{{\dn v} \tl{v} is sometimes called \emph{dento-labial}.}
\footnotetext[4]{The signs for the guttural and labial sibilants have
  become obsolete, and are replaced by the two dots {\dn ,} \tl{ḥ}.}

Unmodified nasal or \tl{anusvāra}, {\dn {\rs -\re}\2} \tl{ṁ} or {\dn \1} \tl{X}.

Unmodified sibilant or \tl{visarga}, {\dn ,} \tl{ḥ}.

Students should be cautioned against using the Roman letters instead of
the Devanāgarī when beginning to learn Sanskrit. The paradigms should be
impressed on the memory in their real and native form, otherwise their
first impressions will become unsettled and indistinct. After some
progress has been made in mastering the grammar and in reading Sanskrit,
the Roman alphabet may be used safely and with advantage.

\s There are fifty letters in the Devanāgarī alphabet: thirty-seven
consonants and thirteen vowels, representing every sound of the Sanskrit
language.

\s One letter, the long {\dn \319w} \tl{\d{\={l}}}, is merely a grammatical
invention; it never occurs in the spoken language.

\s Two sounds, the guttural and labial sibilants, are now without
distinctive representatives in the Devanāgarī alphabet. They are called
\tl{jihāmūlīya}, the tongue-root sibilant, formed near the base of the
tongue; and \tl{upadhmānīya}, i.e. \emph{afflandus}, the labial
sibilant. They are said to have been represented by the signs XXX
(called \tl{vajrākṛti}, having the shape of the thunderbolt) and XXX
(called \tl{gajakumbhākṛti}, having the shape of an elephant's two
frontal bones). (See Vopadeva's \emph{Sanskrit Grammar}, i 18;
\emph{History of Ancient Sanskrit Literature}, p. 508.) Sometimes the
sign XXX, called \tl{ardhavisarga}, \eng{half-visarga}, is used for
both. But in common writing these two signs are now replaced by the two
dots, the \tl{dvivindu}, {\dn ,}, (\tl{dvi}, two, \tl{vindu}, dot) properly
the sign of the unmodified visarga.

\s There are five distinct letters for the five nasals, {\dn \qq{R}} \tl{ṅ},
{\dn \qq{\31Aw}} \tl{ñ}, {\dn \qq{Z}} \tl{ṇ}, {\dn \qq{n}} \tl{n}, {\dn \qq{m}} \tl{m}, as there were
originally five distinct signs for the five sibilants. When, in the
middle of words, these nasals are followed by consonants of their own
class, (\tl{ṅ} by \tl{k}, \tl{kh}, \tl{g}, \tl{gh}; \tl{ñ} by \tl{c},
\tl{ch}, \tl{j}, \tl{jh}; \tl{ṇ} by \tl{ṭ}, \tl{ṭh}, \tl{ḍ}, \tl{ḍh};
\tl{n} by \tl{t}, \tl{th}, \tl{d}, \tl{dh}; \tl{m} by \tl{p}, \tl{ph},
\tl{b}, \tl{bh},) they are often, for the sake of more expeditious
writing, replaced by the dot, which is properly the sign of the
unmodified nasal or anusvāra. Thus we find

\begin{quote}
  {\dn a\2EktA} instead of {\dn aE\317wktA} \tl{aṅkitā}\\
  {\dn a\2EctA} instead of {\dn aE\3D1wtA} \tl{añcitā}\\
  {\dn \7{k}\2EXtA} instead of {\dn \7{k}E\317wXtA} \tl{kuṇḍitā}\\
  {\dn n\2EdtA} instead of {\dn nE\306wdtA} \tl{nanditā}\\
  {\dn k\2EptA} instead of {\dn kEMptA} \tl{kampitā}\\
\end{quote}

The pronunciation remains unaffected by this style of writing.
{\dn a\2EktA} must be pronounced as if it were written {\dn aE\317wktA}
\tl{aṅkitā}, \&c.

The same applies to final {\dn \qq{m}} \tl{m} at the end of a sentence. This
too, though frequently written and printed with the dot above the line,
is to be pronounced as \tl{m}. {\dn ah\2}, \eng{I}, is to be pronounced
like {\dn ah\qq{m}} \tl{aham}. (See preface to \emph{Hitopadeśa}, in M.\ M.'s
\emph{Handbooks for the Study of Sanskrit}, p. viii.)

% TODO finish sanskrit quote
\begin{note}
  Note—According to the Kaumāras final {\dn \qq{m}} \tl{m} \emph{in pausā} may
  be pronounced as anusvāra; cf. \emph{Sārasvatīprakriyā}, ed. Bombay,
  1829,\footnote{This edition, which has lately been reprinted, contains
    the text—ascribed either to Vāṇī herself, i.e. Sarasvatī, the
    goddess of speech (MS Bodl. 386), or to Anubhūti-svarūpa-āchārya,
    whoever that may be—and a commentary. The commentary printed in the
    Bombay editions is called {\dn mhFGrF}, or in MS Bodl. 382
    {\dn m\4dAsF}, i.e. {\dn mhFdAsF}. In MS Bodl. 382 Mahīdhara or
    Mahīdāsabhaṭṭa is said to have written the \emph{Sārasvata} in order
    that his children might read it, and to please Īśa, \eng{the Lord}.
    The date given is 1634, the place Benares, (Śivarājadhanī.)} pp. 12
  and 13. {\dn kOmArA} The Kaumāras are the followers of Kumāra, the reputed
  author of the Kātantra or Kalāpa grammar. (See Colebrooke,
  \emph{Sanskrit Grammar}, preface; and page 315, note.) Śarvavarman is
  sometimes quoted by mistake as the author of this grammar, and an
  unnecessary distinction is made between the Kaumāras and the followers
  of the Kalāpa grammar.
\end{note}

\s Besides the five nasal letters, expressing the nasal sound as
modified by guttural, palatal, lingual, dental, and labial
pronunciation, there are still three nasalized letters, the {\dn \qq{y}\1},
{\dn \qq{l}\1}, {\dn \qq{v}\1}, or {\dn \qq{y}\2}, {\dn \qq{l}\2}, {\dn \qq{v}\2}, XXX, XXX, XXX, which are used to
represent a final {\dn \qq{m}} \tl{m}, if followed by an initial {\dn \qq{y}} \tl{y},
{\dn \qq{l}} \tl{l}, {\dn \qq{v}} \tl{v}, and modified by the pronunciation of these
three semivowels.

% TODO not correct sanskrit
\begin{tabbing}
  Thus \=instead of {\dn t\2 yAEt} \tl{taṁ yāti} we may write {\dn t\305w\qq{y}\1aAEt}
  \tl{XXX};\\
  \>instead of {\dn t\2 lBt\?} \tl{taṁ labhate} we may write
  {\dn t\qq{\3A5w}\1aBt\?} \tl{XXX};\\
  \>instead of {\dn t\2 vhEt} \tl{taṁ vahati} we may write {\dn t\qq{\3A8w}\1ahEt}
  \tl{XXX}.\\
\end{tabbing}

\noindent Or in composition,

% TODO not correct sanskrit
\begin{quote}
  {\dn s\2yAn\2} \tl{saṁyānaṁ} or {\dn s\305w\qq{y}\1aAn\2} \tl{XXX};\\
  {\dn s\2lND\2} \tl{saṁlabdhaṁ} or {\dn s\qq{\3A5w}\1aND\2} \tl{XXX};\\
  {\dn s\2vhEt} \tl{saṁvahati} or {\dn s\qq{\3A8w}\1ahEt} \tl{XXX}.\\
\end{quote}

\s The only consonants which have no corresponding nasals are {\dn \qq{r}}
\tl{r}, {\dn \qq{f}} \tl{ś}, {\dn \qq{q}} \tl{ṣ}, {\dn \qq{s}} \tl{s}, {\dn \qq{h}} \tl{h}. A final
{\dn \qq{m}} \tl{m}, therefore, before any of these letters at the beginning of
words can only be represented by the neutral or unmodified nasal, the
anusvāra.

\begin{tabbing}
  \hspace*{1cm}\={\dn t\2 r\322wEt} \tl{taṁ rakṣati}.\hspace*{1cm}Or in
  composition, \={\dn s\2r\322wEt} \tl{saṁrakṣati}.\\
  \>{\dn t\2 \9{f}ZoEt} \tl{taṁ śṛṇoti}. \>{\dn s\2\9{f}ZoEt} \tl{saṁśṛṇoti}.\\
  \>{\dn t\2 qkAr\2} \tl{taṁ ṣakāraṁ}. \>{\dn s\2\3A4wFvEt}
  \tl{saṁṣṭhīvati}.\\
  \>{\dn t\2 srEt} \tl{taṁ sarati}. \> {\dn s\2srEt} \tl{saṁsarati}.\\
  \>{\dn t\2 hrEt} \tl{taṁ harati}. \> {\dn s\2hrEt} \tl{saṁharati}.\\
\end{tabbing}

\s In the body of a word the only letters which can be preceded by
anusvāra are {\dn \qq{f}} \tl{ś}, {\dn \qq{q}} \tl{ṣ}, {\dn \qq{s}} \tl{s}, {\dn \qq{h}} \tl{h}. Thus
{\dn a\2f,} \tl{aṁśaḥ}, {\dn D\8{n}\2Eq} \tl{dhanūṁṣi}, {\dn yfA\2Es}
\tl{yaśāṁsi}, {\dn Es\2h,} \tl{siṁhaḥ}. Before the semivowels {\dn \qq{y}} \tl{y},
{\dn \qq{r}} \tl{r}, {\dn \qq{l}} \tl{l}, {\dn \qq{v}} \tl{v}, the {\dn \qq{m}} \tl{m} in the body of a
word is never changed into anusvāra. Thus {\dn gMyt\?} \tl{gamyate},
{\dn nm\5,} \tl{namraḥ}, {\dn a\3DDw,} \tl{amlaḥ}. As to {\dn \qq{m}} \tl{m} before
semivowels in the middle of compounds, see \S~9.

\s With the exception of \tl{jihvāmūlīya} XXX XXX (\eng{tongue-root
  letter}), \tl{upadhmānīya} XXX XXX (\eng{to be breathed on}), anusvāra
{\dn {\rs -\re}\2} \tl{ṁ} (\eng{after-sound}), visarga {\dn ,} \tl{ḥ} (\eng{emission},
see Taitt. Brāhm. iii p. 23 a), and \tl{repha r} (\eng{burring}), all
letters are named in Sanskrit by adding \tl{kāra} (\eng{making}) to
their sounds. Thus {\dn a} \tl{a} is called {\dn akAr,} \tl{akāraḥ}; {\dn k}
\tl{ka}, {\dn kkAr,} \tl{kakāraḥ}.

\s The vowels, if initial, are written {\dn a} \tl{a}, {\dn aA} \tl{ā}, {\dn i}
\tl{i}, {\dn I} \tl{ī}, {\dn \31Bw} \tl{ṛ}, {\dn \311w} \tl{\d{\={r}}}, {\dn \318w} \tl{ḷ},
({\dn \319w} \tl{\d{\={l}}}), {\dn u} \tl{u}, {\dn U} \tl{ū}, {\dn e} \tl{e}, {\dn e\?}
\tl{ai}, {\dn ao} \tl{o}, {\dn aO} \tl{au}; if they follow a consonant, they are
written with the following signs: XXX. There is one exception. If the
vowel {\dn \31Bw} \tl{ṛ} follows the consonant {\dn \qq{r}} \tl{r}, it retains its
initial form, and the \tl{r} is written over it. Ex. {\dn En\31Bw\0Et,}
\tl{nirṛtiḥ}.

In certain words which tolerate an hiatus in the body of a word, the
second vowel is written in its initial form. Ex. {\dn go{}ag\5} \tl{goagra},
adj. \eng{preceded by cows}, instead of {\dn go\35Fwg\5} \tl{go'gra} or
{\dn gvAg\5} \tl{gavāgra}; {\dn goa\398w\2} \tl{goaśvaṃ}, \eng{cows and horses};
{\dn \3FEw{}ug} \tl{praüga}, \eng{yoke}; {\dn Ett{}u} \tl{titaü}, \eng{sieve}.
