\def\DevnagVersion{2.17}%@dollars
\chapter{The Alphabet}

\s Sanskrit is properly written with the Devanāgarī alphabet; but the
Bengali, Telugu, and other modern Indian alphabets are commonly employed
for writing Sanskrit in their respective provinces.

\begin{note}
  Note—\tl{Devaṅāgarī} means the \tl{nāgarī} of the gods, or, possibly,
  of the Brāhmans. A more current style of writing, used by Hindus in
  all common transactions where Hindi is the language employed is called
  simply \tl{nāgarī}. Why the alphabet should have been called
  \tl{nāgarī} is unknown. If derived from \tl{nagara}, \eng{city}, it
  might mean the art of writing as first practised in cities (\panini{}
  iv 2, 128). No authority has yet been adduced from any ancient author
  for the employment of the word \tl{Devanāgarī}. In the
  \emph{Lalitavistara} (a life of Buddha, translated from Sanskrit into
  Chinese 76 AD), where a list of alphabets is given, the
  \tl{Devanāgarī} is not mentioned, unless it be intended by the
  \tl{Deva} alphabet. (See \emph{History of Ancient Sanskrit
    Literature}, p.\ 518.) Al-Biruni, in the 11th century, speaks of the
  \tl{Nagara} alphabet as current in Malva (Reinaud, \emph{Mémoire sur
    l'Inde}, p.\ 298).

  No inscriptions have been met with in India anterior to the rise of
  Buddhism. The earliest authentic specimens of writing as the
  inscriptions of king Priyadarśi or Aśoka, about 250 BC. These are
  written in two different alphabets. The alphabet which is found in the
  inscription of Kapurdigiri, and which in the main is the same as that
  of the Arianian coins, is written from right to left. It is clearly of
  Semitic origin, and most closely connected with the Aramaic branch of
  the old Semitic or Phoenician alphabet. The Aramaic letters, however,
  which we know from Egyptian and Palmyrenian inscriptions, have
  experienced further changes since they served as the model for the
  alphabet of Kapurdigiri, and we must have recourse to the more
  primitive types of the ancient Hebrew coins and of the Phoenician
  inscriptions in order to explain some of the letters of the
  Kapurdigiri alphabet.

  But while the transition of the Semitic types into this ancient Indian
  alphabet can be proved with scientific precision, the second Indian
  alphabet, that which is found in the inscription of Girnar, and which
  is the real source of all other Indian alphabets, as well as of those
  of Tibet and Burma, has not as yet been traced back in a satisfactory
  manner to any Semitic prototype (Prinsep's \emph{Indian Antiquities by
  Thomas}, vol.\ ii, p.\ 42). To admit, however, the independent
  invention of a native Indian alphabet is impossible. Alphabets were
  never invented, in the usual sense of that word. They were formed
  gradually, and purely phonetic alphabets always point back to earlier,
  syllabic or ideographic, stages. There are no such traces of the
  growth of an alphabet on Indian soil; and it is to be hoped that new
  discoveries may still bring to light the intermediate links by which
  the alphabet of Girnar, and through it the modern Devanāgarī, may be
  connected with one of the leading Semitic alphabets.
\end{note}

\s Sanskrit is written from left to right.

\begin{note}
  Note—\tl{Saṁskṛta} ({\dn s\2-\9{k}t}) means what is rendered \eng{fit} or
  \tl{perfect}. But \eng{Sanskrit} is not called so because the
  Brāhmans, or still less, because the first Europeans who became
  acquainted with it, considered it the most perfect of all languages.
  \tl{Saṁskṛta} meant what is rendered fit for sacred purposes; hence
  \eng{purified}, \eng{sacred}. A vessel that is purified, a sacrificial
  victim that is properly dressed, a man who has passed through all the
  initiatory rites or \tl{saṁskāras}; all these are called
  \tl{saṁskṛta}. Hence the language which alone was fit for sacred acts,
  the ancient idiom of the Vedas, was called \tl{Saṁskṛta}, or the
  sacred language. The local spoken dialects received the general name
  of \tl{prākṛta}. This did not mean originally \eng{vulgar}, but
  \eng{derived}, \eng{secondary}, \eng{second-rate}, literally `what has
  a source or type,' this source or type (\tl{prakṛti}) being the
  Saṁskṛta or sacred language. (See Vararuci's \emph{Prākṛta-Prakāśa},
  ed.\ Cowell, p.\ xvii.)
\end{note}

\s In writing the Devanāgarī alphabet, the distinctive portion of each
letter is written first, then the perpendicular, and lastly the
horizontal line. Ex. XXX, XXX, {\dn k} \tl{k}; XXX, XXX, {\dn K} \tl{kh};
XXX, XXX, {\dn g} \tl{g}; XXX, XXX, {\dn G} \tl{gh}; XXX, {\dn R} \tl{ṅ}, \&c.

Beginners will find it useful to trace the letters on transparent paper
till they know them well and can write them fluently and correctly.

\s The following are the sounds which are represented in the Devanāgarī
alphabet:
