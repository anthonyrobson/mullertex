\def\DevnagVersion{2.17}%@dollars
\chapter{Rules of Sandhi or the Combination of Letters}

\s In Sanskrit every sentence is considered as one unbroken chain of
syllables. Except where there is a stop, which we should mark by
interpunction, the final letters of each word are made to coalesce with
the initial letters of the following word. This coalescence of final and
initial letters, (of vowels with vowels, of consonants with consonants,
and of consonants with vowels,) is called \emph{Sandhi}.

As certain letters in Sanskrit are incompatible with each other, i.e.\
cannot be pronounced one immediately after the other, they have to be
modified or assimilated in order to faciliate their pronunciation. The
rules, according to which either one or both letters are thus modified,
are called \emph{the rules of Sandhi}.

As according to a general rule the words in a sentence must thus be
glued together, the mere absence of Sandhi is in many cases sufficient
to mark the stops which we have to mark in English by interpunction. Ex.
{\dn a-(vE`nmAhA(My\2 i\2\qb{d}-\7{t} d\?vAnA\2 mh\381wm,}
\tl{astvagnimāhātmyaṃ, indrastu devānāṃ mahattamaḥ}, \eng{Let there be
  the greatness of Agni; nevertheless Indra is the greatest of the
  gods}.

\section{Distinction between External and Internal Sandhi}

\s It is essential, in order to avoid confusion, to distinguish between
the rules of Sandhi which determine the changes of final and initial
letters of words (\tl{padas}), and between those other rules of Sandhi
which apply to the final letters of verbal roots (\tl{dhātu}) and
nominal bases (\tl{prātipadika}) when followed by certain terminations
or suffixes. Though both are based on the same phonetic principles and
are sometimes identical, their application is different. For shortness'
sake it will be best to apply the name of \emph{External Sandhi} %
% second ed
or \emph{Pada Sandhi} %keep space
%
to the changes which take place at the meeting of final and initial
letters of words, and that of \emph{Internal Sandhi} to the changes
produced by the meeting of radical and formative elements.

The rules which apply to final and initial letters of words (\tl{padas})
apply, with few exceptions, to the final and initial letters of the
component parts of compounds, and likewise to the final letters of
nominal bases (\tl{prātipadika}) when followed by the so-called
\emph{Pada}-terminations ({\dn <yA\2} \tl{bhyāṃ}, {\dn EB,} \tl{bhiḥ},
{\dn <y,} \tl{bhyaḥ}, {\dn \7{s}} \tl{su}), or by secondary (\tl{taddhita})
suffixes beginning with any consonants except {\dn \qq{y}}~\tl{y}.

The changes produced by the contact of incompatible letters in the body
of a word should properly be treated under the heads of declension,
conjugation, and derivation. In many cases it is far easier to remember
the words ready-made from the dictionary, or the grammatical paradigms
from the grammar, than to acquire the complicated rules with their
numerous exceptions which are generally detailed in Sanskrit grammars
under the head of Sandhi. It is easier to learn that the participle
passive of {\dn El\qq{h}} \tl{lih}, \eng{to lick}, is {\dn lFY,} \tl{līḍhaḥ},
than to remember the rules according to which {\dn \qq{h}} + {\dn \qq{t}} \tl{h + t} are
changed into {\dn \qq{Y}} + {\dn \qq{t}} \tl{ḍh + t}, {\dn \qq{X}} + {\dn \qq{D}} \tl{ḍ + dh}, and
{\dn \qq{X}} + {\dn \qq{Y}} \tl{ḍ + ḍh}; {\dn \qq{X}} \tl{ḍ} is dropt and the vowel
lengthened; while in {\dn pEr\9{v}\qq{h}} + {\dn t,} \tl{parivṛh + taḥ}, the vowel,
under the same circumstances, remains short: \tl{parivṛh + taḥ} =
\tl{parivṛḍh + taḥ}, \tl{parivṛḍ + dhaḥ} = \tl{parivṛḍ + ḍhaḥ} =
\tl{parivṛḍhaḥ}. In Greek and Latin no rules are given with regard to
changes of this kind. If they are to be given at all in Sanskrit
grammars, they should, to avoid confusion, be kept perfectly distinct
from the rules affecting the final and initial letters of words as
brought together in one and the same sentence.

\section{Classification of Vowels}

\s Vowels are divided into short (\tl{hrasva}), long (\tl{dīrgha}), and
protracted (\tl{pluta}) vowels. Short vowels have one measure
(\tl{mātrā}), long vowels two, protracted vowels three %
% second ed
\panini{1.2.27}%
%
. A consonant is said to last half the time of a short vowel.

\begin{enumerate}
\item Short vowels: {\dn a} \tl{a}, {\dn i} \tl{i}, {\dn u} \tl{u}, {\dn \31Bw} \tl{ṛ},
  {\dn \318w} \tl{ḷ}.
\item Long vowels: {\dn aA} \tl{ā}, {\dn I} \tl{ī}, {\dn U} \tl{ū}, {\dn \311w}
  \tl{\d{\={r}}}, {\dn e} \tl{e}, {\dn e\?} \tl{ai}, {\dn ao} \tl{o}, {\dn aO} \tl{au}.
\item Protracted vowels are indicated by the figure {\dn \rn{3}} \tl{3}; {\dn a\rn{3}}
  \tl{a3}, {\dn aA\rn{3}} \tl{ā3}, {\dn i\rn{3}} \tl{i3}, {\dn I\rn{3}} \tl{ī3}, {\dn e\rn{3}} \tl{e3},
  {\dn aO\rn{3}} \tl{au3}. Sometimes we find {\dn a\rn{3}i} \tl{a3i}, instead of {\dn e\rn{3}}
  \tl{e3}; or {\dn aA\rn{3}u} \tl{ā3u}, instead of {\dn aO\rn{3}} \tl{au3}.
\end{enumerate}

\s Vowels are likewise divided into

\begin{enumerate}
\item Monophthongs (\tl{samānāk.sara}): {\dn a} \tl{a}, {\dn aA} \tl{ā}, {\dn i}
  \tl{i}, {\dn I} \tl{ī}, {\dn u} \tl{u}, {\dn U} \tl{ū}, {\dn \31Bw} \tl{ṛ}, {\dn \311w}
  \tl{\d{\={r}}}, {\dn \318w} \tl{ḷ}.
\item Diphthongs (\tl{sandhyak.sara}): {\dn e} \tl{e}, {\dn e\?} \tl{ai}, {\dn ao}
  \tl{o}, {\dn aO} \tl{au}.
\end{enumerate}

\s All vowels are liable to be nasalized, or to become \tl{anunāsika}:
{\dn a\1}~\tl{XXX}, {\dn aA\1}~\tl{XXX}.

\s Vowels are again divided into light (\tl{laghu}) and heavy
(\tl{guru}). This division is important for metrical purposes.

\begin{enumerate}
\item Light vowels are {\dn a} \tl{a}, {\dn i} \tl{i}, {\dn u} \tl{u}, {\dn \31Bw} \tl{ṛ},
  {\dn \318w} \tl{ḷ}, if not followed by a double consonant.
\item Heavy vowels are {\dn e} \tl{e}, {\dn e\?} \tl{ai}, {\dn ao} \tl{o}, {\dn aO}
  \tl{au}, and any short vowel, if followed by more than one consonant.
\end{enumerate}

\s Vowels are, lastly, divided according to accent, into \emph{acute}
(\tl{udātta}), \emph{grave} (\tl{anudātta}), and \emph{circumflexed}
(\tl{svarita}). The acute vowels are pronounced with a raised tone, the
grave vowels with a low, the circumflexed with an even tone %
% second ed
\panini{1.2.29--32}
%
. Accents are marked in Vedic literature only.

\section{Guṇa and Vṛddhi}

\s Guṇa is the strengthening of {\dn i}~\tl{i}, {\dn I}~\tl{ī}, {\dn u}~\tl{u},
{\dn U}~\tl{ū}, {\dn \31Bw}~\tl{ṛ}, {\dn \311w}~\tl{\d{\={r}}}, {\dn \318w}~\tl{ḷ}, by means of
a preceding {\dn a}~\tl{a}, which raises {\dn i}~\tl{i} and {\dn I}~\tl{ī} to
{\dn e}~\tl{e}, {\dn u}~\tl{u} and {\dn U}~\tl{ū} to {\dn ao}~\tl{o}, {\dn \31Bw}~\tl{ṛ} and
{\dn \311w}~\tl{\d{\={r}}} to {\dn a\qq{r}}~\tl{ar}, {\dn \318w}~\tl{ḷ} to {\dn a\qq{l}}~\tl{al} %
% second ed
\panini{1.1.1}
%
.

By a repetition of the same process the Vṛddhi (\eng{increase}) vowels
are formed, viz.\ {\dn e\?}~\tl{ai} instead of {\dn e}~\tl{e}, {\dn aO}~\tl{au}
instead of {\dn ao}~\tl{o}, {\dn aA\qq{r}}~\tl{ār} instead of {\dn a\qq{r}}~\tl{ar}, and
{\dn aA\qq{l}}~\tl{āl} instead of {\dn a\qq{l}}~\tl{al} %
% second ed
\panini{1.1.1}
%
.

Vowels are thus divided again into:

\begin{tabular}[h]{llllll}
  1.~Simple vowels: & {\dn a}\tl{a}, {\dn aA}\tl{ā}, & {\dn i}\tl{i}, {\dn I}\tl{ī}, &
                                                                        {\dn u}\tl{u},
                                                                        {\dn U}\tl{ū},
  & {\dn \31Bw}\tl{ṛ}, {\dn \311w}\tl{\d{\={r}}}, & {\dn \318w}\tl{ḷ}.\\
  2.~Guṇa vowels: & — & {\dn e}\tl{e} (\tl{a} + \tl{i}), & {\dn ao}\tl{o} (\tl{a}
                                                       + \tl{u}), &
                                                                    {\dn a\qq{r}}\tl{ar},
                  & {\dn a\qq{l}}\tl{al}.\\
  3.~Vṛddhi vowels: & {\dn aA}\tl{ā} & {\dn e\?}\tl{āi} (\tl{a} + \tl{a} +
                                   \tl{i}), & {\dn aO}\tl{āu} (\tl{a} +
                                              \tl{a} + \tl{u}), &
                                                                  {\dn aA\qq{r}}\tl{ār},
                  & {\dn aA\qq{l}}\tl{āl}.\\
\end{tabular}

\s {\dn a}~\tl{a} and {\dn aA}~\tl{ā} do not take guṇa, or, as other grammarians
say, remain unchanged after taking guṇa. Thus in the first person
singular of the reduplicated perfect, which requires guṇa or vṛddhi,
{\dn h\qq{n}} \tl{han} forms with guṇa {\dn jGn} \tl{jaghana}, or with vṛddhi
{\dn jGAn} \tl{jaghāna}, \eng{I have killed}.

\section{Combination of Vowels at the End and Beginning of Words}

\s As a general rule, Sanskrit allows of no hiatus (\tl{vivṛtti}) in a
sentence. If a word ends in a vowel, and the next word begins with a
vowel, certain modifications take place in order to remove the hiatus.

\s For the purpose of explaining the combination of vowels, they may be
divided into two classes:

\begin{enumerate}
\item Those which are liable to be changed into semivowels, {\dn i}~\tl{i},
  {\dn I}~\tl{ī}, {\dn u}~\tl{u}, {\dn U}~\tl{ū}, {\dn \31Bw}~\tl{ṛ},
  {\dn \311w}~\tl{\d{\={r}}}; also the diphthongs, {\dn e}~\tl{e}, {\dn e\?}~\tl{ai},
  {\dn ao}~\tl{o}, {\dn aO}~\tl{au}.

\item Those which are not, {\dn a}~\tl{a}, {\dn aA}~\tl{ā}.
\end{enumerate}

Calling the former liquid,\footnote{The \emph{Prātiśākhya} calls thems
  \tl{nāmin}, for a different reason; see \emph{Rig-veda-prātiśākhya},
  ed.\ M.\ M., p.\ xxiii.} the latter hard vowels, we may say: If the
same vowel (long or short) occurs at the end and beginning of words, the
result is the long vowel \panini{6.1.101}. Thus

\begin{quote}
  {\dn a} or {\dn aA} + {\dn a} or {\dn aA} = {\dn aA} \tl{\v{\={a}}} + \tl{\v{\={a}}} =
  \tl{ā}.\\
  {\dn i} or {\dn I} + {\dn i} or {\dn I} = {\dn I} \tl{\v{\={i}}} + \tl{\v{\={i}}} =
  \tl{ī}.\\
  {\dn u} or {\dn U} + {\dn u} or {\dn U} = {\dn U} \tl{\v{\={u}}} + \tl{\v{\={u}}} =
  \tl{ū}.\\
  {\dn \31Bw} or {\dn \311w} + {\dn \31Bw} or {\dn \311w} = {\dn \311w} \tl{\v{\={ṛ}}} + \tl{\v{\={ṛ}}} =
  \tl{\d{\={r}}}.\footnote{The letter {\dn \318w}~\tl{ḷ} is left out, because
    it is of no practical utility. It is treated like {\dn \31Bw}~\tl{ṛ}, only
    substituting {\dn \qq{l}}~\tl{l} for {\dn \qq{r}}~\tl{r} in guṇa and vṛddhi. Thus
    {\dn \318w} + {\dn a\7{n}b\2D,} \tl{ḷ} + \tl{anubandhaḥ} becomes
    {\dn l\7{n}b\2D,} \tl{lanubandhaḥ}, i.e.\ having \tl{ḷ} as indicatory
    letter.}
\end{quote}

\begin{tabbing}
  Ex.\ \={\dn tv i\2\qb{d},} = {\dn tv\?{\qva}\qb{d},} \tl{tava} + \tl{indraḥ} =
  \tl{tavendraḥ}, \eng{thine is Indra}.\\
  \>{\dn sA u\3C6wA} = {\dn so\3C6wA} \tl{sā} + \tl{uktvā} = \tl{soktvā},
  \eng{she having spoken}.\\
  \>{\dn sA \31BwE\388w,} = {\dn sE\388w\0,} \tl{sā} + \tl{ṛddhiḥ} = \tl{sarddhiḥ},
  \eng{this wealth}.\footnotemark\\
  \>{\dn tv \318wkAr,} = {\dn tvSkAr,} \tl{tava} + \tl{ḷkāraḥ} =
  \tl{tavalkāraḥ}, \eng{thy letter} \tl{ḷ}.
\end{tabbing}

\footnotetext{Some grammarians consider the Sandhi of \tl{\v{\={a}}}
  with \tl{ṛ} optional, but they require the shortening of the long
  \tl{ā}. Ex.\ {\dn b\5\39CwA} + {\dn \31BwEq,} \tl{brahmā} + \tl{ṛṣiḥ} =
  {\dn b\5\39CwEq\0,} \tl{brahmarṣiḥ} or {\dn b\5\39Cw \31BwEq,} \tl{brahma ṛṣiḥ},
  \eng{Brahmā}, a \eng{Rishi}.}

Or in compounds, {\dn kAMy} + {\dn iE\3A3w,} = {\dn kAMy\?E\3A3w,} \tl{kāmya} +
\tl{iṣṭiḥ} = \tl{kāmyeṣṭiḥ}, \eng{an offering for a certain boon}.
{\dn Eht} + {\dn upd\?f,} = {\dn Ehtopd\?f,} \tl{hita} + \tl{upadeśaḥ} =
\tl{hitopadeśaḥ}, \eng{good advice}.

\s If hard vowels (long or short) occur at the end of a word, and the
next begins with a diphthong, the result is vṛddhi \panini{6.1.88}. Thus

\begin{quote}
  {\dn a} or {\dn aA} + {\dn e} = {\dn e\?} \tl{\v{\={a}}} + \tl{e} = \tl{āi}.\\
  {\dn a} or {\dn aA} + {\dn e\?} = {\dn e\?} \tl{\v{\={a}}} + \tl{āi} = \tl{āi}.\\
  {\dn a} or {\dn aA} + {\dn ao} = {\dn aO} \tl{\v{\={a}}} + \tl{o} = \tl{āu}.\\
  {\dn a} or {\dn aA} + {\dn aO} = {\dn aO} \tl{\v{\={a}}} + \tl{āu} = \tl{āu}.
\end{quote}

\begin{tabbing}
  Ex.\ \={\dn tv ev} = {\dn tv\4v} \tl{tava} + \tl{eva} = \tl{tavaiva},
  \eng{of thee only}.\\
  \>{\dn sA e\?E\322w\3A3w} = {\dn s\4E\322w\3A3w} \tl{sā} + \tl{aikṣiṣṭa} =
  \tl{saikṣiṣṭa}, \eng{she saw}.\\
  \>{\dn tv ao\3A4w,} = {\dn tvO\3A4w,} \tl{tava} + \tl{oṣṭhaḥ} =
  \tl{tavauṣṭhaḥ}, \eng{thy lip}.\\
  \>{\dn sA aO(\7{s}\3C8wvtF} = {\dn sO(\7{s}\3C8wvtF} \tl{sā} + \tl{autsukyavatī} =
  \tl{sautsukyavatī}, \eng{she desirous}.
\end{tabbing}

Or in compounds, {\dn rAm} + {\dn e\?\398wy{\rdt}} = {\dn rAm\4\398wy{\rdt}} \tl{rāma} +
\tl{aiśvaryaṃ} = \tl{rāmaiśvaryaṁ}, \eng{the lordship of Rāma}. {\dn sFtA}
+ {\dn aOpMy\2} = {\dn sFtOpMy\2} \tl{sītā} + \tl{aupamyaṃ} =
\tl{sītaupamyaṃ}, \eng{similarly with Sītā}, the wife of Rāma.

\s If a simple liquid vowel (long or short) occurs at the end of a word,
and the next begins with any vowel or diphthong, the result is change of
the liquid vowel into a semivowel \panini{6.1.77}. Thus

\begin{tabular}[h]{rlrl}
  \ldelim\{{5}{*}[{\dn i} or {\dn I}] & {\dn a} or {\dn aA} = {\dn y} or {\dn yA} &
                                                               \ldelim\{{5}{*}[\tl{\u{\={i}}}]
  & \tl{\u{\={a}}} = \tl{y\u{\={a}}}.\\
  & {\dn \31Bw} or {\dn \311w} = {\dn \9{y}} or {\dn \qx{y}} & & \tl{ṛ\u{\={i}}} =
                                      \tl{yṛ\u{\={i}}}.\\
                               & {\dn u} or {\dn U} = {\dn \7{y}} or {\dn \8{y}} & & \tl{\u{\={u}}} = \tl{y\u{\={u}}}.\\
  & {\dn e} or {\dn e\?} = {\dn y\?} or {\dn y\4} & & \tl{e}, \tl{ai} = \tl{ye},
                                    \tl{yai}.\\
  & {\dn ao} or {\dn aO} = {\dn yo} or {\dn yO} & & \tl{o}, \tl{au} = \tl{yo},
                                    \tl{yau}.\\
  \ldelim\{{5}{*}[{\dn \31Bw} or {\dn \311w}] & {\dn a} or {\dn aA} = {\dn r} or {\dn rA} &
                                                                \ldelim\{{5}{*}[\tl{\v{\={ṛ}}}]
  & \tl{\u{\={a}}} = \tl{r\u{\={a}}}.\\
                               & {\dn i} or {\dn I} = {\dn Er} or {\dn rF} & & \tl{\u{\={i}}} = \tl{r\u{\={i}}}.\\
                               & {\dn u} or {\dn U} = {\dn z} or {\dn !} & & \tl{\u{\={u}}} = \tl{r\u{\={u}}}.\\
  & {\dn e} or {\dn e\?} = {\dn r\?} or {\dn r\4} & & \tl{e}, \tl{ai} = \tl{re},
                                    \tl{rai}.\\
  & {\dn ao} or {\dn aO} = {\dn ro} or {\dn rO} & & \tl{o}, \tl{au} = \tl{ro},
                                    \tl{rau}.\\
  \ldelim\{{5}{*}[{\dn u} or {\dn U}] & {\dn a} or {\dn aA} = {\dn v} or {\dn vA} &
                                                               \ldelim\{{5}{*}[\tl{\u{\={u}}}]
  & \tl{\u{\={a}}} = \tl{v\u{\={a}}}.\\
                               & {\dn i} or {\dn I} = {\dn Ev} or {\dn vF} & & \tl{\u{\={i}}} = \tl{v\u{\={i}}}.\\
  & {\dn \31Bw} or {\dn \311w} = {\dn \9{v}} or {\dn \qx{v}} & & \tl{r\u{\={i}}} =
                                      \tl{vṛ\u{\={i}}}.\\
  & {\dn e} or {\dn e\?} = {\dn v\?} or {\dn v\4} & & \tl{e}, \tl{ai} = \tl{ve},
                                    \tl{vai}.\\
  & {\dn ao} or {\dn aO} = {\dn vo} or {\dn vO} & & \tl{o}, \tl{au} = \tl{vo}, \tl{vau}.\\
\end{tabular}

\begin{tabbing}
  Ex.\ \={\dn dED a/} = {\dn d@y/} \tl{dadhi} + \tl{atra} =
  \tl{dadhyatra}, \eng{milk here}.\\
  \>{\dn k\9{t}\0 ut} = {\dn k\7{/}\0t} \tl{kartṛ} + \tl{uta} = \tl{kartruta},
  \eng{doing moreover}.\\
  \>{\dn m\7{D} iv} = {\dn mE@vv} \tl{madhu} + \tl{iva} = \tl{madhviva},
  \eng{like honey}.\\
  \>{\dn ndF e\?X-y} = {\dn n\38Dw\4X-y} \tl{nadī} + \tl{aiḍasya} =
  \tl{nadyaiḍasya}, \eng{the river of Aiḍa}.
\end{tabbing}

In compounds, {\dn ndF} + {\dn aT{\rdt}} = {\dn n\38DwT{\rdt}} \tl{nadī} +
\tl{arthaṁ} = \tl{nadyarthaṁ}, \eng{for the sake of a river}.

\begin{note}
  Note—Some native grammarians allow, except in compounds, the omission
  of this Sandhi, but they require in that case that a long final vowel
  be shortened. Ex.\ {\dn c\387wF a/} \tl{cakrī atra} may be {\dn c\3E7w/}
  \tl{cakryatra} or {\dn cE\387w a/} \tl{cakri atra}.
\end{note}

\s If a guṇa vowel occurs at the end of a word, and the next begins with
any vowel or diphthong (except \tl{\u{\={a}}}), the last element of the
guṇa vowel is changed into a semivowel. If \tl{ă} follows, \tl{ă} is
elided, and no change takes place in the dipthong; see \S\,41
\panini{6.1.78}. Thus

\begin{quote}
  {\dn e} (\tl{e}) + any vowel (except \tl{ă}) = {\dn a\qq{y}} (\tl{ay}).\\
  {\dn ao} (\tl{o}) + any vowel (except \tl{ă}) = {\dn a\qq{v}} (\tl{av}).\\
\end{quote}

\begin{tabbing}
  Ex.\ \={\dn sK\? aAgQC} = {\dn sKyAgQC} \tl{sakhe āgaccha} =
  \tl{sakhayāgaccha}, \eng{Friend, come!}\\
  \>{\dn sK\? ih} = {\dn sKEyh} \tl{sakhe iha} = \tl{sakhayiha},
  \eng{Friend, here!}\\
  \>{\dn \3FEwBo eEh} = {\dn \3FEwBv\?Eh} \tl{prabho ehi} = \tl{prabhavehi},
  \eng{Lord, come near!}\\
  \>{\dn \3FEwBo aOqD\2} = {\dn \3FEwBvOqD\2} \tl{prabho auṣadhaṁ} =
  \tl{prabhavauṣadhaṁ}, \eng{Lord, medicine}.
\end{tabbing}

In compounds, {\dn go} + {\dn If,} = {\dn gvFf,} \tl{go} + \tl{īśaḥ} =
\tl{gavīśaḥ}. There are various exceptions in compounds where {\dn go}
\tl{go} is treated as {\dn gv} \tl{gava} (\S\,41).

\s If a vṛddhi vowel occurs at the end of a word, and the next begins
with any vowel or diphthong, the last element is changed into a
semivowel \panini{6.1.78}. Thus

\begin{quote}
  {\dn e\?} (\tl{ai}) + any vowel = {\dn aA\qq{y}} (\tl{āy}).\\
  {\dn aO} (\tl{au}) + any vowel = {\dn aA\qq{v}} (\tl{āv}).\\
\end{quote}

\begin{tabbing}
  Ex.\ \={\dn E\399wy\4 aT\0,} = {\dn E\399wyAyT\0,} \tl{śriyai arthaḥ} =
  \tl{śriyāyarthaḥ}.\\
  \>{\dn E\399wy\4 \31Bwt\?} = {\dn E\399wyA\9{y}t\?} \tl{śriyai ṛte} = \tl{śriyāyṛte}.\\
  \>{\dn rvO a-tEmt\?} = {\dn rvAv-tEmt\?} \tl{ravau astamite} =
  \tl{ravāvastamite}, \eng{after sunset}.\\
  \>{\dn tO iEt} = {\dn tAEvEt} \tl{tau iti} = \tl{tāviti}.
\end{tabbing}

In composition, {\dn nO} + {\dn aT{\rdt}} = {\dn nAvT{\rdt}} \tl{nau} + \tl{arthaṁ}
= \tl{nāvarthaṁ}, \eng{for the sake of ships}.

\s These two rules, however, are liable to certain modifications:

\begin{enumerate}
\item The final {\dn \qq{y}}~\tl{y} and {\dn \qq{v}}~\tl{v} of {\dn a\qq{y}}~\tl{ay},
  {\dn a\qq{v}}~\tl{av}, which stand according to rule for {\dn e}~\tl{e},
  {\dn ao}~\tl{o}, may be dropt before all vowels (except \tl{ă}, \S\,41);
  not, however, in composition. Thus most MSS and printed editions
  change

  \begin{quote}
    {\dn sK\? aAgQC} \tl{sakhe āgaccha}, not into {\dn sKyAgQC}
    \tl{sakhayāgaccha}, but into {\dn sK aAgQC} \tl{sakha āgaccha}.\\
    {\dn sK\? ih} \tl{sakhe iha}, not into {\dn sKEyh} \tl{sakhayiha}, but
    into {\dn sK ih} \tl{sakha iha}.\\
    {\dn \3FEwBo eEh} \tl{prabho ehi}, not into {\dn \3FEwBv\?Eh} \tl{prabhavehi},
    but into {\dn \3FEwB eEh} \tl{prabha ehi}.\\
    {\dn \3FEwBo aOqD\2} \tl{prabho auṣadhaṁ}, not into
    {\dn \3FEwBvOqD\2} \tl{prabhavauṣadhaṁ}, but into {\dn \3FEwB
    aOqD\2} \tl{prabha auṣadhaṁ}.
  \end{quote}

\item The final {\dn \qq{y}}~\tl{y} of {\dn aA\qq{y}}~\tl{āy}, which stands for
  {\dn e\?}~\tl{āi}, may be dropt before all vowels, and it is usual to drop
  it in our editions. Thus

  \begin{quote}
    {\dn E\399wy\4 aT\0,} \tl{śriyai arthaḥ} is more usually written
    {\dn E\399wyA aT\0,} \tl{śriyā arthaḥ} instead of {\dn E\399wyAyT\0,}
    \tl{śriyāyarthaḥ}.
  \end{quote}

\item The final {\dn \qq{v}}~\tl{v} or {\dn aA\qq{v}}~\tl{āv}, for {\dn aO}~\tl{āu}, may be
  dropt before all vowels, but is more usually retained in our editions.
  Thus

  \begin{quote}
    {\dn tO iEt} \tl{tau iti} is more usually written {\dn tAEvEt}
    \tl{tāviti}, and not {\dn tA iEt} \tl{tā iti}.
  \end{quote}
\end{enumerate}

\begin{note}
  Note—Before the particle {\dn u}~\tl{u} the dropping of the final
  {\dn \qq{y}}~\tl{y} and {\dn \qq{v}}~\tl{v} is obligatory.

  It is without any reason that the final {\dn \qq{y}}~\tl{y} of guṇa and vṛddhi
  and the final {\dn \qq{v}}~\tl{v} of guṇa are generally dropt, while the final
  {\dn \qq{v}}~\tl{v} of vṛddhi is generally retained. It would be more
  consistent either always to retain the final semivowels or always to
  drop them. See \emph{Rig-veda-prātiśākhya}, ed.\ M.\ M., sūtras 129,
  132, 135; Pāṇini 6.1.78; 8.3.19.
\end{note}

\s In all these cases the hiatus, occasioned by the dropping of
{\dn \qq{y}}~\tl{y} and {\dn \qq{v}}~\tl{v}, remains, and the rules of Sandhi are not to
be applied again.

\s {\dn e}~\tl{e} and {\dn ao}~\tl{o}, before short {\dn a}~\tl{a}, remain unchanged,
and the initial {\dn a}~\tl{a} is elided \panini{6.1.109}.

\begin{tabbing}
  Ex.\ \={\dn Efv\? a/} = {\dn Efv\?\35Fw/} \tl{śive atra} = \tl{śive'tra},
  \eng{in Śiva there}.\\
  \>{\dn \3FEwBo a\7{n}\9{g}hAZ} = {\dn \3FEwBo\35Fw\7{n}\9{g}hAZ} \tl{prabho anugṛhāṇa}
  = \tl{prabho'nugṛhāṇa}, \eng{Lord, please}.
\end{tabbing}

In composition this elision is optional \panini{6.1.122}.

\begin{tabbing}
  Ex.\ \={\dn go} + {\dn a\398wA,} = {\dn go\35Fw\398wA,} or {\dn go{}a\398wA,} \tl{go} +
  \tl{aśvāḥ} = \tl{go'śvāḥ} or \tl{go aśvāḥ}, \eng{cows and horses}.
\end{tabbing}

In some compounds {\dn gv} \tl{gava} must or may be substituted for {\dn go}
\tl{go}, if a vowel follows; {\dn gvA\322w,} \tl{gavākṣaḥ}, \eng{a
  window}, literally \eng{a bull's eye}; {\dn gv\?\306w\qb{d},} \tl{gavendraḥ},
\eng{lord of kine} (a name of Krishna); {\dn gvAEjn\2} or {\dn go\35FwEjn\2}
\tl{gavājinaṁ} or \tl{go'jinaṁ}, \eng{a bull's hide}.

\section{Unchangeable Vowels (\tl{Pragṛhya})}
\s There are certain terminations the final vowels of which are not
liable to any Sandhi rules. These vowels are called \tl{pragṛhya}
\panini{1.1.11} by Sanskrit grammarians. They are,

\begin{enumerate}
\item The terminations of the dual in {\dn I}~\tl{ī}, {\dn U}~\tl{ū}, and
  {\dn e}~\tl{e}, whether of nouns or verbs.

  \begin{tabbing}
    Ex.\ \={\dn kvF imO} \tl{kavī imau}, \eng{these two poets}.\\
    \>{\dn EgrF etO} \tl{girī etau}, \eng{these two hills}.\\
    \>{\dn sA\8{D} imO} \tl{sādhū imau}, \eng{these two merchants}.\\
    \>{\dn b\306w\8{D} aAny} \tl{bandhū ānaya}, \eng{bring the two friends}.\\
    \>{\dn lt\? et\?} \tl{late ete}, \eng{these two creepers}.\\
    \>{\dn Ev\38Dw\? im\?} \tl{vidye ime}, \eng{these two sciences}.\\
    \>{\dn fyAt\? aB\0kO} \tl{śayāte arbhakau}, \eng{the two children
      lie down}.\\
    \>{\dn fyAvh\? aAvA\2} \tl{śayāvahe āvāṁ}, \eng{we two lie down}.\\
    \>{\dn yAc\?t\? aT{\rdt}} \tl{yācete arthaṁ}, \eng{they two ask for
      money}.\\
  \end{tabbing}

  \begin{note}
    Note—Exceptions occur, as {\dn mZFv} \tl{maṇīva}, i.e.\ {\dn mZF iv}
    \tl{maṇī iva}, \eng{like two jewels}; {\dn d\2ptFv} \tl{daṁpatīva},
    i.e.\ {\dn d\2ptF iv} \tl{daṁpatī iva}, \eng{like husband and wife}.
  \end{note}

\item The terminations of {\dn amF} \tl{amī} and {\dn a\8{m}} \tl{amū}, the
  nominative plural masculine and the nominative \emph{dual} of the
  pronoun {\dn ad\qq{s}} \tl{adas} \panini{1.1.12}.

  \begin{tabbing}
    Ex.\ \={\dn amF a\398wA,} \tl{amī aśvāḥ}, \eng{these horses}.\\
    \>{\dn amF iqv,} \tl{amī iṣavaḥ}, \eng{these arrows}.\\
    \>{\dn a\8{m} aB\0kO} \tl{amū arbhakau}, \eng{these two children}. (This
    follows from rule 1.)
  \end{tabbing}
\end{enumerate}

\section{Irregular Sandhi}

\s The following are a few cases of irregular Sandhi which require to be
stated. When a preposition ending in {\dn a} or {\dn aA} \tl{\v{\={a}}} is
followed by a verb beginning with {\dn e}~\tl{e} or {\dn ao}~\tl{o}, the result
of the coalescence of the vowels is {\dn e}~\tl{e} or {\dn ao}~\tl{o}, not
{\dn e\?}~\tl{ai} or {\dn aO}~\tl{au} \panini{6.1.94}.

\begin{tabbing}
  Ex.\ \={\dn \3FEw} + {\dn ejt\?} = {\dn \3FEw\?jt\?} \tl{pra} + \tl{ejate} =
  \tl{prejate}.\\
  \>{\dn up} + {\dn eqt\?} = {\dn up\?qt\?} \tl{upa} + \tl{eṣate} =
  \tl{upeṣate}.\\
  \>{\dn \3FEw} + {\dn eqyEt} = {\dn \3FEw\?qyEt} \tl{pra} + \tl{eṣayati} =
  \tl{preṣayati}.\footnotemark\\
  \>{\dn prA} + {\dn eKEt} = {\dn pr\?KEt} \tl{parā} + \tl{ekhati} =
  \tl{parekhati}.\\
  \>{\dn up} + {\dn aoqEt} = {\dn upoqEt} \tl{upa} + \tl{oṣati} =
  \tl{upoṣati}.\\
  \>{\dn prA} + {\dn aohEt} = {\dn prohEt} \tl{parā} + \tl{ohati} =
  \tl{parohati}.
\end{tabbing}

\footnotetext{In nouns derived from {\dn \3FEw\?\qq{q}} \tl{preṣ}, the rule is
  optional. Ex.\ {\dn \3FEw\?\309wy} or {\dn \3FEw\4\309wy} \tl{preṣya} or \tl{praiṣya},
  \eng{a messenger}. {\dn \3FEw\?q} \tl{preṣa}, \eng{a gleaner}, is derived
  from {\dn \3FEw} \tl{pra} and {\dn I\qq{q}} \tl{īṣ}.}

\begin{note}
  This is not the case before the two verbs {\dn e\qq{D}} \tl{edh}, \eng{to
    grow}, and {\dn i}~\tl{i}, \eng{to go}, if raised by guṇa to {\dn e}~\tl{e}
  \panini{6.1.89}.

  \begin{tabbing}
    Ex.\ \={\dn up} + {\dn eDt\?} = {\dn up\4Dt\?} \tl{upa} + \tl{edhate} =
    \tl{upaidhate}.\\
    \>{\dn av} + {\dn eEt} = {\dn av\4Et} \tl{ava} + \tl{eti} = \tl{avaiti}.
  \end{tabbing}

  In verbs derived from nouns, and beginning with {\dn e} or {\dn ao} \tl{e} or
  \tl{o}, the elision of the final {\dn a} or {\dn aA} \tl{\u{\={a}}} of the
  preposition is optional.
\end{note}

\s If a root beginning with {\dn \31Bw}~\tl{ṛ} is preceded by a preposition
ending in {\dn a}~\tl{a} or {\dn aA}~\tl{ā}, the two vowels coalesce into
{\dn aA\qq{r}}~\tl{ār} instead of {\dn a\qq{r}}~\tl{ar} \panini{6.1.91}.

\begin{tabbing}
  Ex.\ \={\dn ap} + {\dn \31BwQCEt} = {\dn apAQC\0Et} \tl{apa} + \tl{ṛcchati} =
  \tl{apārcchati}.\\
  \>{\dn av} + {\dn \31BwZAEt} = {\dn avAZA\0Et} \tl{ava} + \tl{ṛṇāti} =
  \tl{avārṇāti}.\\
  \>{\dn \3FEw} + {\dn \31Bwjt\?} = {\dn \3FEwAj\0t\?} \tl{pra} + \tl{ṛjate} =
  \tl{prārjate}.\\
  \>{\dn prA} + {\dn \31BwqEt} = {\dn prAq\0Et} \tl{parā} + \tl{ṛṣati} =
  \tl{parārṣati}.
\end{tabbing}

\begin{note}
  In verbs derived from nouns and beginning with {\dn \31Bw}~\tl{ṛ}, this
  lengthening of the {\dn a}~\tl{a} of the preposition is optional
  \panini{6.1.92}.

  In certain compounds {\dn \31BwZ\2} \tl{ṛṇaṁ}, \eng{debt}, and {\dn \31Bwt,}
  \tl{ṛtaḥ}, \eng{affected}, take vṛddhi instead of guṇa if preceded by
  {\dn a}~\tl{a}; {\dn \3FEw} + {\dn \31BwZ\2} = {\dn \3FEwAZ{\rdt}} \tl{pra} + \tl{ṛṇaṁ} =
  \tl{prārṇaṁ}, \eng{principal debt}; {\dn \31BwZ\2} + {\dn \31BwZAZ\2}
  \tl{ṛṇaṁ} + \tl{ṛṇārṇaṁ} = \tl{ṛṇārṇaṁ}, \eng{debt contracted to
    liquidate another debt}; {\dn fok} + {\dn \31Bwt,} = {\dn fokAt\0,}
  \tl{śoka} + \tl{ṛtaḥ} = \tl{śokārtaḥ}, \eng{affected by sorrow}.
  Likewise {\dn U\qq{h}} \tl{ūh}, the substitute for {\dn vA\qq{h}} \tl{vāh},
  \eng{carrying}, forms vṛddhi with a preceding {\dn a}~\tl{a} in a
  compound. Thus {\dn Ev\398w} + {\dn Uh,} \tl{viśva} + \tl{ūhaḥ}, the
  accusative plural of {\dn Ev\398wvA\qq{h}} \tl{viśvavāh}, is {\dn Ev\398wOh,}
  \tl{viśvauhaḥ} \panini{6.1.89, vārt}.
\end{note}

\s If the initial {\dn ao}~\tl{o} in {\dn ao\3A4w,} \tl{oṣṭhaḥ}, \eng{lip}, and
{\dn ao\7{t},} \tl{otuḥ}, \eng{cat}, is preceded in a compound by {\dn a} or {\dn aA}
\tl{\u{\={a}}}, the two vowels may coalesce into {\dn aO}~\tl{au} or
{\dn ao}~\tl{o} \panini{6.1.94, vārt}.

\begin{tabbing}
  Ex.\ \={\dn aDr} + {\dn ao\3A4w,} = {\dn aDrO\3A4w,} or {\dn aDro\3A4w,}
  \tl{adhara} + \tl{oṣṭhaḥ} = \tl{adharauṣṭhaḥ} or \tl{adharoṣṭhaḥ},
  \eng{the lower lip}.\\
  \>{\dn -\8{T}l} + {\dn ao\7{t},} = {\dn -\8{T}lO\7{t},} or {\dn -\8{T}lo\7{t},} \tl{sthūla} +
  \tl{otuḥ} = \tl{sthūlautuḥ} or \tl{sthūlotuḥ}, \eng{a big cat}.
\end{tabbing}

If {\dn ao\3A4w} \tl{oṣṭha} and {\dn ao\7{t}} \tl{otu} are preceded by {\dn a} or {\dn aA}
\tl{\u{\={a}}} in the middle of a sentence, they follow the general
rule.

Ex.\ {\dn mm} + {\dn ao\3A4w,} = {\dn mmO\3A4w,} \tl{mama} + \tl{oṣṭhaḥ} =
\tl{mamauṣṭhaḥ}, \eng{my lip}.

\s As irregular compounds the following are mentioned by native
grammarians:

\begin{quote}
  {\dn -v\4r\2} \tl{svairaṁ}, \eng{wilfulness}, and {\dn -v\4Er\qq{n}}
  \tl{svairin}, \eng{self-willed}, from {\dn -v} + {\dn Ir} \tl{sva} +
  \tl{īra}.\\
  {\dn a\322wOEhZF} \tl{akṣauhiṇī}, \eng{a complete army}, from {\dn a\322w} +
  {\dn UEhnF} \tl{akṣa} + \tl{ūhinī}.\\
  {\dn \3FEwOY,} \tl{prauḍhaḥ}, from {\dn \3FEw} + {\dn UY,} \tl{pra} +
  \tl{ūḍhaḥ}, \eng{full-grown}.\\
  {\dn \3FEwOh,} \tl{prauhaḥ}, \eng{investigation}, from {\dn \3FEw} + {\dn Uh,}
  \tl{pra} + \tl{ūhaḥ}.\\
  {\dn \3FEw\4q,} \tl{praiṣaḥ}, \eng{a certain prayer}, from {\dn \3FEw} +
  {\dn eq,} \tl{pra} + \tl{eṣaḥ}. (See \S\,43.)\\
  {\dn \3FEw\4\309wy,} \tl{praiṣyaḥ}, \eng{a messenger}.
\end{quote}

\s The final {\dn ao}~\tl{o} of indeclinable words is not liable to the rules
of Sandhi \panini{1.1.15}.

Ex.\ {\dn aho ap\?Eh} \tl{aho apehi}, \eng{Halloo, go away!}

\s Indeclinables consisting of a single vowel, with the exception of
{\dn aA}~\tl{ā} (\S\,49), are not liable to the rules of Sandhi
\panini{1.1.14}.

\begin{tabbing}
  Ex.\ \={\dn i i\2\qb{d}} \tl{i indra}, \eng{Oh Indra!} {\dn u um\?f} \tl{u umeśa},
  \eng{Oh lord of Umā!}\\
  \>{\dn aA ev\2} \tl{ā evaṁ}, \eng{Is it so indeed?}
\end{tabbing}

\s If {\dn aA}~\tl{ā} (which is written by Indian grammarians {\dn aA\qq{R}}
\tl{āṅ}) is used as a preposition before verbs, or before nouns in the
sense of `so far as' (inclusively or exclusively) or `a little,' it is
liable to the rules of Sandhi.

\begin{tabbing}
  Ex.\ \={\dn aA a@yynA\qq{t}} = {\dn aA@yynA\qq{t}} \tl{ā adhyayanāt} =
  \tl{ādhyayanāt}, \eng{until the reading begins}.\\
  \>{\dn aA ekd\?fA\qq{t}} = {\dn e\?kd\?fA\qq{t}} \tl{ā ekadeśāt} = \tl{aikadeśāt},
  \eng{to a certain place}.\\
  \>{\dn aA aAloEct\2} = {\dn aAloEct\2} \tl{ā ālocitaṁ} = \tl{ālocitaṁ},
  \eng{regarded a little}.\\
  \>{\dn aA u\309wZ\2} = {\dn ao\309wZ\2} \tl{ā uṣṇaṁ} = \tl{oṣṇaṁ}, \eng{a little
    warm}.\\
  \>{\dn aA iEh} = {\dn eEh} \tl{ā ihi} = \tl{ehi}, \eng{come here}.
\end{tabbing}

If {\dn aA}~\tl{ā} is used as an interjection, it is not liable to Sandhi,
according to \S\,48.

Ex.\ {\dn aA ev\2 Ekl t\qq{t}} \tl{ā, evam kila tat}, \eng{Ah,—now I
  recollect,—it is just so}.

\s Certain particles remain unaffected by Sandhi.

Ex.\ {\dn h\? i\2\qb{d}} \tl{he indra}, \eng{Oh Indra}.

\s A protracted vowel remains unaffected by Sandhi, because it is always
supposed to stand at the end of a sentence \panini{6.1.125; 8.2.82}.

Ex.\ {\dn d\?vd\381wA\rn{3} eEh} \tl{devadattā3 ehi}, \eng{Devadatta, come here!}

% TODO: Complete this table
SANDHI TABLE

%%% Local Variables:
%%% mode: latex
%%% TeX-master: "../main"
%%% End:
