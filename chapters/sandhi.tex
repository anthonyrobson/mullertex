\def\DevnagVersion{2.17}%@dollars
\chapter{Rules of Sandhi or the Combination of Letters.}

\s In Sanskrit every sentence is considered as one unbroken chain of
syllables. Except where there is a stop, which we should mark by
interpunction, the final letters of each word are made to coalesce with
the initial letters of the following word. This coalescence of final and
initial letters, (of vowels with vowels, of consonants with consonants,
and of consonants with vowels,) is called \emph{Sandhi}.

As certain letters in Sanskrit are incompatible with each other, i.e.
cannot be pronounced one immediately after the other, they have to be
modified or assimilated in order to faciliate their pronunciation. The
rules, according to which either one or both letters are thus modified,
are called \emph{the rules of Sandhi}.

As according to a general rule the words in a sentence must thus be
glued together, the mere absence of Sandhi is in many cases sufficient
to mark the stops which we have to mark in English by interpunction. Ex.
{\dn a-(vE`nmAhA(My\2 i\2\qb{d}-\7{t} d\?vAnA\2 mh\381wm,}
\tl{astvagnimāhātmyaṃ, indrastu devānāṃ mahattamaḥ}, \eng{Let there be
  the greatness of Agni; nevertheless Indra is the greatest of the
  gods}.

\section{Distinction between External and Internal Sandhi.}

\s It is essential, in order to avoid confusion, to distinguish between
the rules of Sandhi which determine the changes of final and initial
letters of words (\tl{padas}), and between those other rules of Sandhi
which apply to the final letters of verbal roots (\tl{dhātu}) and
nominal bases (\tl{prātipadika}) when followed by certain terminations
or suffixes. Though both are based on the same phonetic principles and
are sometimes identical, their application is different. For shortness'
sake it will be best to apply the name of \emph{External Sandhi} to the
changes which take place at the meeting of final and initial letters of
words, and that of \emph{Internal Sandhi} to the changes produced by the
meeting of radical and formative elements.

The rules which apply to final and initial letters of words (\tl{padas})
apply, with few exceptions, to the final and initial letters of the
component parts of compounds, and likewise to the final letters of
nominal bases (\tl{prātipadika}) when followed by the so-called
\emph{Pada}-terminations ({\dn <yA\2} \tl{bhyāṃ}, {\dn EB,} \tl{bhiḥ},
{\dn <y,} \tl{bhyaḥ}, {\dn \7{s}} \tl{su}), or by secondary (\tl{taddhita})
suffixes beginning with any consonants except \tl{y}.

The changes produced by the contact of incompatible letters in the body
of a word should properly be treated under the heads of declension,
conjugation, and derivation. In many cases it is far easier to remember
the words ready-made from the dictionary, or the grammatical paradigms
from the grammar, than to acquire the complicated rules with their
numerous exceptions which are generally detailed in Sanskrit grammars
under the head of Sandhi. It is easier to learn that the participle
passive of {\dn El\qq{h}} \tl{lih}, \eng{to lick}, is {\dn lFY,} \tl{līḍhaḥ},
than to remember the rules according to which {\dn \qq{h}} + {\dn \qq{t}} \tl{h + t} are
changed into {\dn \qq{Y}} + {\dn \qq{t}} \tl{ḍh + t}, {\dn \qq{X}} + {\dn \qq{D}} \tl{ḍ + dh}, and
{\dn \qq{X}} + {\dn \qq{Y}} \tl{ḍ + ḍh}; {\dn \qq{X}} \tl{ḍ} is dropt and the vowel
lengthened; while in {\dn pEr\9{v}\qq{h}} + {\dn t,} \tl{parivṛh + taḥ}, the vowel,
under the same circumstances, remains short: \tl{parivṛh + taḥ} =
\tl{parivṛḍh + taḥ}, \tl{parivṛḍ + dhaḥ} = \tl{parivṛḍ + ḍhaḥ} =
\tl{parivṛḍhaḥ}. In Greek and Latin no rules are given with regard to
changes of this kind. If they are to be given at all in Sanskrit
grammars, they should, to avoid confusion, be kept perfectly distinct
from the rules affecting the final and initial letters of words as
brought together in one and the same sentence.

\section{Classification of Vowels.}

\s Vowels are divided into short (\tl{hrasva}), long (\tl{dīrgha}), and
protracted (\tl{pluta}) vowels. Short vowels have one measure
(\tl{mātrā}), long vowels two, protracted vowels three. A consonant is
said to last half the time of a short vowel.

\begin{enumerate}
\item Short vowels: {\dn a} \tl{a}, {\dn i} \tl{i}, {\dn u} \tl{u}, {\dn \31Bw} \tl{ṛ},
  {\dn \318w} \tl{ḷ}.
\item Long vowels: {\dn aA} \tl{ā}, {\dn I} \tl{ī}, {\dn U} \tl{ū}, {\dn \311w}
  \tl{\d{\={r}}}, {\dn e} \tl{e}, {\dn e\?} \tl{ai}, {\dn ao} \tl{o}, {\dn aO} \tl{au}.
\item Protracted vowels are indicated by the figure {\dn \rn{3}} \tl{3}; {\dn a\rn{3}}
  \tl{a3}, {\dn aA\rn{3}} \tl{ā3}, {\dn i\rn{3}} \tl{i3}, {\dn I\rn{3}} \tl{ī3}, {\dn e\rn{3}} \tl{e3},
  {\dn aO\rn{3}} \tl{au3}. Sometimes we find {\dn a\rn{3}i} \tl{a3i}, instead of {\dn e\rn{3}}
  \tl{e3}; or {\dn aA\rn{3}u} \tl{ā3u}, instead of {\dn aO\rn{3}} \tl{au3}.
\end{enumerate}

\s Vowels are likewise divided into

\begin{enumerate}
\item Monophthongs (\tl{samānāk.sara}): {\dn a} \tl{a}, {\dn aA} \tl{ā}, {\dn i}
  \tl{i}, {\dn I} \tl{ī}, {\dn u} \tl{u}, {\dn U} \tl{ū}, {\dn \31Bw} \tl{ṛ}, {\dn \311w}
  \tl{\d{\={r}}}, {\dn \318w} \tl{ḷ}.
\item Diphthongs (\tl{sandhyak.sara}): {\dn e} \tl{e}, {\dn e\?} \tl{ai}, {\dn ao}
  \tl{o}, {\dn aO} \tl{au}.
\end{enumerate}

\s All vowels are liable to be nasalized, or to become \tl{anunāsika}:
{\dn a\1} \tl{XXX}, {\dn aA\1} \tl{XXX}.

\s Vowels are again divided into light (\tl{laghu}) and heavy
(\tl{guru}). This division is important for metrical purposes.

\begin{enumerate}
\item Light vowels are {\dn a} \tl{a}, {\dn i} \tl{i}, {\dn u} \tl{u}, {\dn \31Bw} \tl{ṛ},
  {\dn \318w} \tl{ḷ}, if not followed by a double consonant.
\item Heavy vowels are {\dn e} \tl{e}, {\dn e\?} \tl{ai}, {\dn ao} \tl{o}, {\dn aO}
  \tl{au}, and any short vowel, if followed by more than one consonant.
\end{enumerate}

\s Vowels are, lastly, divided according to accent, into \emph{acute}
(\tl{udātta}), \emph{grave} (\tl{anudātta}), and \emph{circumflexed}
(\tl{svarita}). The acute vowels are pronounced with a raised tone, the
grave vowels with a low, the circumflexed with an even tone. Accents are
marked in Vedic literature only.

\section{Guṇa and Vṛddhi.}

\s Guṇa is the strengthening of {\dn i} \tl{i}, {\dn I} \tl{ī}, {\dn u} \tl{u},
{\dn U} \tl{ū}, {\dn \31Bw} \tl{ṛ}, {\dn \311w} \tl{\d{\={r}}}, {\dn \318w} \tl{ḷ}, by means of
a preceding {\dn a} \tl{a}, which raises {\dn i} \tl{i} and {\dn I} \tl{ī} to {\dn e}
\tl{e}, {\dn u} \tl{u} and {\dn U} \tl{ū} to {\dn ao} \tl{o}, {\dn \31Bw} \tl{ṛ} and {\dn \311w}
\tl{\d{\={r}}} to {\dn a\qq{r}} \tl{ar}, {\dn \318w} \tl{ḷ} to {\dn a\qq{l}} \tl{al}.

By a repetition of the same process the Vṛddhi (\eng{increase}) vowels
are formed, viz. {\dn e\?} \tl{ai} instead of {\dn e} \tl{e}, {\dn aO} \tl{au}
instead of {\dn ao} \tl{o}, {\dn aA\qq{r}} \tl{ār} instead of {\dn a\qq{r}} \tl{ar}, and
{\dn aA\qq{l}} \tl{āl} instead of {\dn a\qq{l}} \tl{al}.

Vowels are thus divided again into

%%% Local Variables:
%%% mode: latex
%%% TeX-master: "../main"
%%% End:
