\def\DevnagVersion{2.17}%@dollars
\chapter{Special and General Tenses and the Ten Classes of Verbs.}

\s Sanskrit grammarians have divided all verbs into ten classes,
according to certain modifications which their roots undergo before the
terminations of the Present, the Imperfect, the Optative, and
Imperative. This division is very useful, and will be retained with some
slight alterations. One and the same root may belong to different
classes. Thus {\dn B\5A\qq{f}} \tl{bhrāś}, {\dn <lA\qq{f}} \tl{bhlāś}, {\dn B\5\qq{m}}
\tl{bhram}, {\dn \387w\qq{m}} \tl{kram}, {\dn \3CAw\qq{m}} \tl{klam}, {\dn /\qq{s}} \tl{tras},
{\dn \7{/}\qq{V}} \tl{truṭ}, {\dn l\qq{q}} \tl{laṣ} belong to the Bhū and Div classes;
{\dn B\5Aft\?} \tl{bhrāśate} or {\dn B\5A\35Bwyt\?} \tl{bhrāśyate}, \&c.
(Pāṇini, iii. 1, 70). Again, {\dn -\7{k}} \tl{sku}, {\dn -t\2\qq{B}} \tl{stambh},
{\dn -\7{t}\2\qq{B}} \tl{stumbh}, {\dn -kM\qq{B}} \tl{skambh}, {\dn -\7{k}M\qq{B}} \tl{skumbh}
belong to the Su and Krī classes; {\dn -\7{k}noEt} \tl{skunoti} or {\dn -\7{k}nAEt}
\tl{skunāti} (Pāṇini, iii. 1, 82).

\s The four tenses and moods which require this modification of the root
will be called the \emph{Special Modified Tenses}; the rest the
\emph{General or Unmodified Tenses}. Thus the root {\dn Ec} \tl{ci} is
changed in the Present, Imperfect, Optative, and Imperative into {\dn Ec\7{n}}
\tl{cinu}. Hence {\dn Ec\7{n}m,} \tl{cinumaḥ}, \eng{we search}; {\dn aEc\7{n}m}
\tl{acinuma}, \eng{we searched}. But the Past Participle {\dn Ect,}
\tl{citaḥ}, \eng{searched}, or the Reduplicated Perfect {\dn EcQ\7{y},}
\tl{cicyuḥ}, \eng{they have searched}, without the {\dn \7{n}} \tl{nu}. We call
{\dn Ec} \tl{ci}, the root, {\dn Ec\7{n}} \tl{cinu}, the base of the special
tenses.

\s Verbal bases are first divided into two divisions:

\begin{enumerate}
\item Bases which in the modified tenses end in {\dn a} \tl{a}.
\item Bases which in the modified tenses end in any letter but {\dn a} \tl{a}.
\end{enumerate}

This second division is subdivided into,

\begin{enumerate}
\item Bases which insert {\dn \7{n}} \tl{nu}, {\dn u} \tl{u}, or {\dn nF} \tl{nī},
  between the root and the terminations.
\item Bases which take the terminations without any intermediate
  element.
\end{enumerate}

\section{I. First Division.}

\s The first division comprises four classes:

\begin{enumerate}
\item The Bhū class (the first with the native grammarians, and called
  by them {\dn <vAEd} \tl{bhvādi}, the because the first verb in their
  lists is {\dn \8{B}} \tl{bhū}, \eng{to be}).

  \begin{enumerate}
  \item {\dn a} \tl{a} is added to the last letter of the root.
  \item The vowel of the root takes guṇa, where possible (i.e. long or
    short \tl{i}, \tl{u}, \tl{ṛ}, if final; short \tl{i}, \tl{u},
    \tl{ṛ}, \tl{ḷ}, if followed by \emph{one} consonant).

    {\dn \7{b}\qq{D}} \tl{budh}, \eng{to know}; {\dn boDEt} \tl{bodh-a-ti}, \eng{he
      knows}. {\dn \8{B}} \tl{bhū}, \eng{to be}; {\dn BvEt} \tl{bhav-a-ti},
    \eng{he is}.
  \end{enumerate}

  \begin{note}
    Note—The accent in verbs of the Bhū class was originally (as we know
    from the ancient Vedic language) on the radical vowel; hence guṇa of
    that vowel.

    Many derivative verbs follow this class; such as causatives,
    {\dn BAvyEt} \tl{bhāvayati}, \eng{he causes to be}; desideratives,
    {\dn \7{b}\8{B}qEt} \tl{bubhūṣati}, \eng{he wishes to be}, from {\dn \8{B}}
    \tl{bhū}; intensives in the ātmanepada, {\dn b\?EB\38Dwt\?}
    \tl{bebhidyate}, \eng{he cuts much}; and denominatives,
    {\dn loEhtAyEt} \tl{lohitāyati}, \eng{he grows red}.
  \end{note}

\item The Tud class (the sixth class with native grammarians, and called
  by them {\dn \7{t}dAEd} \tl{tudādi}, because the first root in their lists
  is {\dn \7{t}\qq{d}} \tl{tud}, \eng{to strike}).

  \begin{enumerate}
  \item {\dn a} \tl{a} is added to the last letter of the root.
  \item Before this {\dn a} \tl{a}, final {\dn i} \tl{i} and {\dn I} \tl{ī} are
    changed to {\dn i\qq{y}} \tl{iy}; {\dn u} \tl{u} and {\dn U} \tl{ū} to {\dn u\qq{v}}
    \tl{uv}; {\dn \31Bw} \tl{ṛ} to {\dn Er\qq{y}} \tl{riy}; {\dn \311w} \tl{\d{\={r}}} to
    {\dn i\qq{r}} \tl{ir} (\S~110).

    {\dn \7{t}\qq{d}} \tl{tud}, \eng{to strike}; {\dn \7{t}dEt} \tl{tud-a-ti}. {\dn Er}
    \tl{ri}, \eng{to go}; {\dn EryEt} \tl{riy-a-ti}. {\dn \8{n}} \tl{nū}, \eng{to
    praise}; {\dn \7{n}vEt} \tl{nuv-a-ti}. {\dn \9{m}} \tl{mṛ}, \eng{to die};
  {\dn Em\5yt\?} \tl{mriy-a-te}. {\dn \qx{k}} \tl{k\d{\={r}}}, \eng{to scatter};
  {\dn EkrEt} \tl{kir-a-ti}.
  \end{enumerate}

  \begin{note}
    Note—The accent in verbs of the Tud class was originally on the
    intermediate {\dn a} \tl{a}; hence never guṇa of the radical vowel.
  \end{note}

\item The Div class (the fourth with native grammarians, and called by
  them {\dn EdvAEd} \tl{divādi}, because the first root in their lists is
  {\dn Ed\qq{v}} \tl{div}, \eng{to play}).

  \begin{enumerate}
  \item {\dn y} \tl{ya} is added to the last letter of the root.

    {\dn n\qq{h}} \tl{nah}, \eng{to bind}; {\dn n\39DwEt} \tl{nah-ya-ti}. {\dn \7{b}\qq{D}}
    \tl{budh}, \eng{to awake}; {\dn \7{b}@yt\?} \tl{budh-ya-te}.
  \end{enumerate}

  \begin{note}
    Note—The accent in verbs of the Div class is now on the radical
    vowel; but there are traces to show that some verbs of this class
    had the accent originally on {\dn y} \tl{ya}.
  \end{note}

\item The Cur class (the tenth with native grammarians, and called by
  them {\dn \7{c}rAEd} \tl{curādi}, because the first root in their lists is
  {\dn \7{c}\qq{r}} \tl{cur}, \eng{to steal}).

  \begin{enumerate}
  \item {\dn ay} \tl{aya} is added to the last letter of the root.
  \item If the root ends in a simple consonant, preceded by {\dn a} \tl{a},
    {\dn a} \tl{a} is lengthened to {\dn aA} \tl{ā}. {\dn d\qq{l}} \tl{dal}, \eng{to
      cut}; {\dn dAlyEt} \tl{dāl-aya-ti} (many exceptions).
  \item If the root ends in a simple consonant, preceded by {\dn i} \tl{i},
    {\dn u} \tl{u}, {\dn \31Bw} \tl{ṛ}, {\dn \318w} \tl{ḷ}, these vowels take guṇa, while
    \tl{\d{\={r}}} becomes {\dn I\qq{r}} \tl{īr}. {\dn E\39Aw\qq{q}} \tl{śliṣ}, \eng{to
      embrace}; {\dn \39Aw\?qyEt} \tl{śleṣ-aya-ti}. {\dn \7{c}\qq{r}} \tl{cur}, \eng{to
      steal}; {\dn coryEt} \tl{cor-aya-ti}. {\dn \9{m}\qq{q}} \tl{mṛṣ}, \eng{to
      endure}; {\dn mq\0yt\?} \tl{marṣ-aya-te}. {\dn \qx{k}\qq{t}} \tl{k\d{\={r}}t},
    \eng{to praise}; {\dn kFt\0yEt} \tl{kīrt-aya-ti}.
  \item Final {\dn i} \tl{i}, {\dn I} \tl{ī}, {\dn u} \tl{u}, {\dn U} \tl{ū}, {\dn \31Bw}
    \tl{ṛ}, and {\dn \311w} \tl{\d{\={r}}}, take vṛddhi. {\dn E\385w} \tl{jri},
    \eng{to grow old}; {\dn \385wAyyEt} \tl{jrāy-aya-ti}. {\dn mF} \tl{mī},
    \eng{to walk}; {\dn mAyyEt} \tl{māy-aya-ti}. {\dn \9{D}} \tl{dhṛ}, \eng{to
      hold}; {\dn DAryEt} \tl{dhār-aya-ti}. {\dn \qx{p}} \tl{p\d{\={r}}},
    \eng{to fill}; {\dn pAryEt} \tl{pār-aya-ti}.
  \end{enumerate}

  \begin{note}
    Note—Many, if not all roots arranged under this class by native
    grammarians, are secondary roots, and identical in form with
    causatives, denominatives, \&c. This class differs from other
    classes, inasmuch as verbs belonging to it, keep their modificatory
    syllable {\dn ay} \tl{aya} throughout, in the unmodified as well as in
    the modified tenses, except in the benedictive parasmaipada. The
    accent was on the first {\dn a} \tl{a} of {\dn ay} \tl{aya}.
  \end{note}
\end{enumerate}

\section{II. Second Division.}

\s The second division comprises all verbs which do not, in the special
tenses, end in {\dn a} \tl{a} before the terminations.

It is a distinguishing feature of this second division that, before
certain terminations, all verbs belonging to it require strengthening of
their radical vowel, or if they take {\dn \7{n}} \tl{nu}, {\dn u} \tl{u}, {\dn nF}
\tl{nī}, strengthening of the vowels of these syllables. This
strengthening generally takes place by means of guṇa, but {\dn nF} \tl{nī}
is raised to {\dn nA} \tl{nā} in the Krī, and {\dn \qq{n}} \tl{n} to {\dn n} \tl{na}
in the Rudh class.

We shall call the terminations which require strengthening of the
inflective base the \emph{weak terminations}, and the base before them
the \emph{strong base}; and vice versa, the terminations which do not
require strengthening of the base the \emph{strong terminations}, and
the base before them the \emph{weak base}.

Originally the accent fell on the strong terminations, and on the strong
base, thus establishing throughout an equilibrium between base and
termination.

\subsection{a. Bases which take {\dn \7{n}} \tl{nu}, {\dn u} \tl{u}, {\dn nF}
  \tl{nī}.}

\s This first subdivision comprises three classes:

\begin{enumerate}
\item The Su class (the fifth class with native grammarians, and called
  by them {\dn -vAEd} \tl{svādi}, because the first root in their lists is
  {\dn \7{s}} \tl{su}).

  \begin{enumerate}
  \item {\dn \7{n}} \tl{nu} is added to the last letter of the root before
    strong terminations; {\dn no} \tl{no} before weak terminations.

    Ex. {\dn \7{s}} \tl{su}, \eng{to squeeze out}; {\dn \7{s}\7{n}m,} \tl{su-nu-máḥ}, 1st
    person plural present. {\dn \7{s}noEm} \tl{su-nó-mi}, 1st person singular
    present.
  \end{enumerate}

\item The Tan class (the eighth class with native grammarians, and
  called by them {\dn tnAEd} \tl{tanādi}, because the first root in their
  lists is {\dn t\qq{n}} \tl{tan}).

  \begin{enumerate}
  \item {\dn u} \tl{u} is added to the last letter of the root before
    strong terminations; {\dn ao} \tl{o} before weak terminations.

    Ex. {\dn t\qq{n}} \tl{tan}, \eng{to stretch}; {\dn t\7{n}m,} \tl{tan-u-máḥ},
    1st person plural present. {\dn tnoEm} \tl{tan-ó-mi}, 1st person
    singular present.
  \end{enumerate}

  \begin{note}
    Note—All verbs belonging to this class end in {\dn \qq{n}} \tl{n}, except
    one, {\dn \9{k}} \tl{kṛ}, {\dn kroEm} \tl{karomi}, \eng{I do}.
  \end{note}

\item The Krī class (the ninth with native grammarians, and called by
  them {\dn \3E7wAEd} \tl{kryādi}, because the first root in their lists is
  {\dn \387wF} \tl{krī}).

  \begin{enumerate}
  \item {\dn nF} \tl{nī} is added to the last letter of the root before
    strong terminations; {\dn nA} \tl{nā} before weak terminations; {\dn n}
    \tl{na} before strong terminations beginning with vowels.

    Ex. {\dn \387wF} \tl{krī}, \eng{to buy}; {\dn \387wFZFm,} \tl{krī-ṇī-máḥ},
    1st person plural present. {\dn \387wFZAEm} \tl{krī-ṇā-mi}, 1st person
    singular present. {\dn \387wFZE\306wt} \tl{krī-ṇ-ánti}, 3rd person plural
    present.
  \end{enumerate}
\end{enumerate}

\subsection{b. Bases to which the terminations are joined immediately.}

\s The second division comprises three classes:

\begin{enumerate}
\item The Ad class (the second class with native grammarians, and called
  by them {\dn adAEd} \tl{adādi}, because the first root in their lists is
  {\dn a\qq{d}} \tl{ad}, \eng{to eat}).

  \begin{enumerate}
  \item The terminations are added immediately to the last letter of the
    base; and in the contact of vowels with vowels, vowels with
    consonants, consonants with vowels, and consonants with consonants,
    the phonetic rules explained above (\S\S~107–145) must be carefully
    observed.

  \item The strong base before the weak terminations takes guṇa where
    possible (\S~296, 1, 6).

    Ex. {\dn El\qq{h}} \tl{lih}, \eng{to lick}; {\dn El\39Cw,} \tl{lih-máḥ}, \eng{we
      lick}. {\dn l\?E\39Cw} \tl{léh-mi}, \eng{I lick}. {\dn l\?E\322w} \tl{lek-ṣi},
    \eng{thou lickest} (\S~127). {\dn lFY} \tl{līḍha}, \eng{you lick}
    (\S~128). {\dn al\?\qq{V}} \tl{aleṭ}, \eng{thou lickedst} (\S~128).
  \end{enumerate}

  The intensive verbs conjugated in the parasmaipada follow this class.
\end{enumerate}
%%% Local Variables:
%%% mode: latex
%%% TeX-master: "../main"
%%% End:
